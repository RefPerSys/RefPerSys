% file IntelliSys2021RefPerSys/refpersys-IntelliSys2021.tex
%%% https://saiconference.com/IntelliSys2021/CallforPapers#Guidelines

%!  Papers must be in English and not more than 20 pages in length.
%!  Paper content must be original and relevant to one of the many conference topics.
%!  Authors are required to ensure accuracy of quotations, citations, diagrams, maps, and tables.
%!  Figures and tables need to be placed where they are to appear in the text and must be clear and easy to view.
%!  Papers must follow format according to the downloadable template
%!
%!  Review Process
%!
%!  The review process will be double-blind.  Therefore, please anonymize
%!  your submission. This means that all submissions must contain no
%!  information identifying the author(s) or their organization(s): Do not
%!  put the author(s) names or affiliation(s) at the start of the paper,
%!  anonymize citations to and mentions of your own prior work that are
%!  directly related to your present work, and do not include funding or
%!  other acknowledgments.

\documentclass{svproc}
% to typeset URLs, URIs, and DOIs
\usepackage{url}
\usepackage{relsize}
\def\UrlFont{\rmfamily}

\bibliographystyle{spbasic}

\begin{document}

\input{generated-intellisys-commands}

\newcommand{\RefPerSys}[0]{{\textit{\textsc{RefPerSys}}}}

\mainmatter              % start of a contribution
%
\title{Design and Roadmap for a Reflexive Persistent System for
  Symbolic Artificial Intelligence} \titlerunning{\RefPerSys}
\author{First Author \and Second author} \authorrunning{RefPerSys
  authors}

\institute{RefPerSys informal team}

\maketitle

\begin{abstract}
  The \RefPerSys{} software is a reflexive persistent software project, under development, for Linux.
\end{abstract}

\textbf{tracks}: expert systems; agents and multi-agent systems.

\section{introduction}

A \textsc{\textbf{Ref}}lexive \textsc{\textbf{Per}}sistent software
\textsc{\textbf{Sys}}tem can be extended to combine some ``expert
system'' inference engine, with some rules and meta-rules, reified as
mutable objects in a persistent memory heap. We describe in this paper
some ongoing work, and major design ideas, about a free software
symbolic artificial intelligence prototype system targetted for Linux
workstations. A major source of inspiration for \RefPerSys{} are the
\textsc{Malice} then \textsc{Caia} software systems
described in
\cite{Pitrat:1996:FGCS, Pitrat:2009:AST,  Pitrat:2009:ArtifBeings}

\bibliography{bib-refpersys}

%%% for a draft that should be acceptable
\begin{flushright}
  \begin{relsize}{-1}
    Our draft \texttt{git} ID is \texttt{\textit{\rpsgitcommit}}. \\
    It was generated on \texttt{\rpsdate}.
  \end{relsize}
\end{flushright}

\end{document}

%%% For emacs:
%%%%%%%%%%%%%%%%%%%%%%%%%%%%%%%%%%%%%%%%%%%%%%%%%%%%%%%%%%%%%%%%
%% Local Variables: ;;
%% compile-command: "./make-paper.sh" ;;
%% End: ;;
%%%%%%%%%%%%%%%%%%%%%%%%%%%%%%%%%%%%%%%%%%%%%%%%%%%%%%%%%%%%%%%%
