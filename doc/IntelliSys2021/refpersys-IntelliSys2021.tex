% file IntelliSys2021RefPerSys/refpersys-IntelliSys2021.tex
%%% https://saiconference.com/IntelliSys2021/CallforPapers#Guidelines

%!  Papers must be in English and not more than 20 pages in length.
%!  Paper content must be original and relevant to one of the many conference topics.
%!  Authors are required to ensure accuracy of quotations, citations, diagrams, maps, and tables.
%!  Figures and tables need to be placed where they are to appear in the text and must be clear and easy to view.
%!  Papers must follow format according to the downloadable template
%!
%!  Review Process
%!
%!  The review process will be double-blind.  Therefore, please anonymize
%!  your submission. This means that all submissions must contain no
%!  information identifying the author(s) or their organization(s): Do not
%!  put the author(s) names or affiliation(s) at the start of the paper,
%!  anonymize citations to and mentions of your own prior work that are
%!  directly related to your present work, and do not include funding or
%!  other acknowledgments.

\documentclass{svproc}
% to typeset URLs, URIs, and DOIs
\usepackage{url}
\usepackage{relsize}
\def\UrlFont{\rmfamily}

\bibliographystyle{spbasic}

\begin{document}

\input{generated-intellisys-commands}

\newcommand{\RefPerSys}[0]{{\textit{\textsc{RefPerSys}}}}

\mainmatter              % start of a contribution
%
\title{Design and Roadmap for a Reflexive Persistent System for
  Symbolic Artificial Intelligence} \titlerunning{\RefPerSys}
\author{First Author \and Second author} \authorrunning{RefPerSys
  authors}

\institute{RefPerSys informal team}

\maketitle

\begin{abstract}
  The \RefPerSys{} software is a reflexive persistent software
  project, with an open source (GPLv3+) license, under development,
  for Linux workstation.
\end{abstract}

\textbf{tracks}: expert systems; agents and multi-agent systems.

\section{introduction}

A \textsc{\textbf{Ref}}lexive \textsc{\textbf{Per}}sistent software
\textsc{\textbf{Sys}}tem can be extended to combine some ``expert
system'' inference engine, with some rules and meta-rules, reified as
mutable objects in a persistent memory heap. We describe in this paper
some ongoing work, and major design ideas, about \RefPerSys, a free
software symbolic artificial intelligence prototype system targetted
for Linux workstations. A major source of inspiration for that system
are the \textsc{Malice} then \textsc{Caia} software systems described
in \cite{Pitrat:1996:FGCS}, \cite{Pitrat:2009:AST},
\cite{Pitrat:2009:ArtifBeings}. We are aware that open source systems
\cite{Weber:2004:SuccessOpenSource} are essential to the success of
any risky software project, like \RefPerSys.

\subsection{user features needed by artificial intelligence software systems}

\subsubsection{Data persistence:}

A symbolic artificial intelligence system could run for a long time,
perhaps hours or even days, interacting with its users, as some Linux
process. But its user would interact with that system, then be taken
by other professional activities (writing or reviewing scientific
papers, finding funding, interacting with students, or managers, or
clients, organizing seminars, ... teaching, etc...).

Since the data of such a system could be ported to other computers,
after some hardware upgrade, or partially shared with other users
(e.g. students or colleagues), it is important that the data format
used by such a system is well documented and self-describable. Since
computers are varied in endianness and word-size, it is preferable to
use some documented textual format, and be open to existing libraries
or frameworks, including database servers like \textsc{PostGreSQL},
\textsc{MongoDB} etc.... \cite{Date:2005:Database-in-Depth}

Given the popularity of distributed versioning systems like
\texttt{git} - generally used to manage human-written source code in
some textual format - and availability of remote services above them,
it make senses to use some ``textual form'', inspired by programming
languages or de facto formats like \textsc{Xml} or \textsc{Json}. The
persistent data can then be viewed as some domain specific language
\cite{Fowler:2010:domain} \cite{Starynkevitch-DSL2011} and also as
some ``source code'', as soon as some software modules -and
eventually, machine code in plugins or shared libraries
\cite{drepper:2011:write-shared-lib} - can be generated from that same
data.

The approach for keeping data should be guided by principles of
\textbf{orthogonal persistence} \cite{dearle:2009:orthogonal}. And
garbage collection algorithms \cite{jones:2016:gchandbook} are
relevant for keeping or discarding such data. Obvious some data, such
as data describing the current state of the user interface, should not
be kept in files; we call such data the \textbf{transient data} of
\RefPerSys.

Using textual data format (visible in ordinary editors like
\textsc{Gnu Emacs}) has the advantage of easing the debugging of such
systems: for the human developers working on \RefPerSys, looking into
generated textual files (in C++ or some other programming language),
for generated code; in textual formats inspired by \textit{Json} for
data. For similar software engineering reasons, many compilers,
including GCC, are still generating assembler code and not directly
binary object files.

\subsubsection{terminology related to data:} Following \cite{abadi:2012:theory-objects},
\cite{hindley:2008:lambda}, \cite{queinnec:2003:lisp}, we call
``\textbf{objects}'' the data (and their abstract data types) which is
mutable and changing during runtime, and ``\textbf{immutable values}''
the data which never changes during runtime (but could be detected as
useless and later removed or destroyed by the garbage collector). When
we speak of \textbf{\RefPerSys{} values}, we mean either mutable
objects or immutables values (including boxed strings, tagged
integers, closures, set of objects, etc...).

\subsubsection{Concurrency:}
Since current computers, including cheap laptops, have multi-core
processors in 2021, it is important to process data in parallel, using
several cores, in particular to gain performance. Most Linux computers
today can run multi-threaded applications, thanks to their operating
system \cite{Arpaci-Dusseau:2018:OSBook}. Some hardware can even be
programmed in specialized programming languages like \textsc{OpenCL}
\cite{kaeli:2015:heterogeneous-opencl}. The $\pi$-calculus
\cite{sangiorgi:2003:pi-calculus} and directed algebraic topology
\cite{fajstrup:2016:directed} are then relevant as a theoretical
framework to reason about parallelism. If access to cloud computing
resources is available, running several \RefPerSys{} processes on
different nodes of some datacenter could be considered. In practice,
multi-threaded programming \cite{butenhof:1997:programming} is
important.

\subsubsection{Meta-programming and dynamic code generation:}
During execution, \RefPerSys{} should optimize its behavior (both to
the particular instance of the problem it is handling, and perhaps to
the actual hardware it is running on) by using partial evaluation
techniques \cite{futamura:1999:partial-evaluation}. Any human expert
software developer, when challenged with solving some real-life
problem, will benchmark various different source code and choose the
``best'' one. In some cases, hardware differences (size of CPU cache,
disk speed, number of cores, network speed, reliability of network
connections - Wifi or Ethernet) may justify a partial rewrite of
performance critical routines.

\subsubsection{Ergonomical user-interface:} As soon as we want to make
a software system which is usable by non-expert users, some graphical
user interface is practically needed, since in 2021 most computer
users have one or a few color screens, some computer mouse, and some
keyboard\footnote{The particular keyboard layout - e.g. AZERTY layout
in France, QWERTY layout in the USA is managed by existing operating
system layers. The screen size varies also greatly, and requires some
runtime adaptation of the sotware to its user.}. However, most
keyboards have ``function keys'', and each individual user should
assign some of them to common queries. Each human user of \RefPerSys{}
obviously have his/her own preferences about colors, fonts, keyboard
and mouse usage, and these user preferences should be given then
managed by the system. Auto-completion facilities are useful for
readability reasons, and this is why software developers are choosing
names for the public interface of their software library very
carefully. The choice between a Web interface and desktop graphical
interface (e.g. using the \textsc{Qt}, \textsc{Fltk}, or \textsc{Gtk}
widget toolkits, or building above the \textsc{Gnu} \texttt{emacs}
editor) has a practical importance, for hand-written code. But once
\RefPerSys{} is capable of generating C++ code, it should in principle
be able to change the user interface more easily. For example, the
\textsc{Qt} GUI toolkit, and the \textsc{Wt} web toolkit\footnote{See
respectively {\texttt{qt.io}} and {\texttt{www.webtoolkit.eu/wt}} ...}
have very similar APIs.

\subsubsection{Rule-based knowledge systems:} A knowledge based system, or an ``expert system''
like \RefPerSys{} should contain many rules for its inference
engine. Each rule $\rho$ of the form $C_1, C_2, ... C_n \Rightarrow
\alpha_1, \alpha_2 ... \alpha_p$ and is interpreted as: \textbf{when}
every condition $C_i$ is true \textbf{then} do the associated actions
$\alpha_j$ in the order given.  Their conditions $C_i$ should be
\textit{unordered} and contain variables, so $\rho$ should behave like
the rule $\rho'$ written as $C_n, C_{n-1}, ... C_1 \Rightarrow
\alpha_1, \alpha_2 ... \alpha_p$.  The set of rules should in
principle be non-ordered, and in practice their order should not
matter much.  Notice that similar rules are used to document and
formalize -for humans- some programming language semantics, see
\cite{pierce:2002:types} and \cite{pierce:2005:advanced}.

Of course, at the machine level, the conditions of some given rule are
represented in some data structure somehow ordered in the computer
memory, but that order should not be relevant for inference: the
overall behavior of the system should not change much when rules are
reloaded in memory in a different order than in some previous
execution of the \RefPerSys{} system.

\bibliography{bib-refpersys}

%%% for a draft that should be acceptable
\begin{flushright}
  \begin{relsize}{-1}
    Our draft \texttt{git} ID is \texttt{\textit{\rpsgitcommit}}. \\
    It was generated on \texttt{\rpsdate}.
  \end{relsize}
\end{flushright}

\end{document}

%%% For emacs:
%%%%%%%%%%%%%%%%%%%%%%%%%%%%%%%%%%%%%%%%%%%%%%%%%%%%%%%%%%%%%%%%
%% Local Variables: ;;
%% compile-command: "./make-paper.sh" ;;
%% End: ;;
%%%%%%%%%%%%%%%%%%%%%%%%%%%%%%%%%%%%%%%%%%%%%%%%%%%%%%%%%%%%%%%%
